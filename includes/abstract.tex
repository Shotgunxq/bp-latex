V tejto bakalárskej práci sa zaoberáme vývojom webovej aplikácie s trojvrstvovou architektúrou zameranej na e-learning matematiky, konkrétne pravdepodobnosti. 
Cieľom práce bolo navrhnúť a implementovať užívateľsky orientovaný frontend pomocou Angular frameworku. 
Backend aplikácie bol vyvinutý pomocou Node.js a frameworku Express.js s cieľom poskytnúť efektívne spracovanie dát a logiky aplikácie. 
Na ukladanie a spracovanie užívateľských dát a obsahu bola použitá databáza PostgreSQL. 
Celá aplikácia je nasadená v Docker kontajneroch, čo umožňuje jednoduchšiu distribúciu a nasadenie aplikácie. 
Výsledkom je komplexná e-learningová platforma, ktorá umožňuje študentom testovať svoje znalosti prostredníctvom testov, úloh a študijných materiálov, a tiež analyzovať ich pokrok a vývoj. 
Tento projekt predstavuje dôležitý krok smerom k moderným pedagogickým metódam, ktoré využívajú technologické inovácie na zlepšenie vzdelávania.