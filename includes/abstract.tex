V tejto bakalárskej práci sa zaoberáme vývojom trojvrstvovej webovej aplikácie
zameranej na e-learning matematiky, konkrétne pravdepodobnosti a štatistiky. Cieľom
práce bolo navrhnúť a implementovať užívateľsky orientovaný frontend pomocou
Angular frameworku, pričom sú využívané knižnice Bootstrap a Material UI na
zabezpečenie intuitívneho rozhrania. Na druhej strane, backend aplikácie bol vyvinutý
pomocou Node.js a frameworku Next.js s cieľom poskytnúť efektívne spracovanie dát
a logiky aplikácie. S PostgreSQL databázou sme pracovali na ukladaní a spracovaní
užívateľských dát a obsahu. Celá aplikácia je nakoniec nasadená v Docker
kontajneroch, čo umožňuje jednoduchšiu distribúciu a nasadenie aplikácie. Výsledkom
je komplexná e-learningová platforma, ktorá umožňuje študentom testovať svoje
znalosti prostredníctvom testov, úloh a študijných materiálov, a tiež analyzovať ich
pokrok a vývoj. Tento projekt predstavuje dôležitý krok smerom k moderným
pedagogickým metódam, ktoré využívajú technologické inovácie na zlepšenie
vzdelávania.