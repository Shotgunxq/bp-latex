K úspešnému spusteniu webovej aplikácie je potrebné mať na vašom zariadení nainštalované nasledujúce:

\begin{itemize}
    \item \textbf{Windows Subsystem for Linux}

    Inštalácia prebieha pomocou Windows PowerShell, kde je potrebné zadať príkaz:
    \begin{verbatim}
    wsl --install
    \end{verbatim}
    Po automatickej inštalácii bude potrebné vytvoriť si lokálne konto pre tento subsystém. Pokračujte podľa pokynov inštalátora.

    \item \textbf{Docker Desktop}

    Na inštaláciu Docker Desktop je potrebné stiahnuť inštalátor zo stránky \url{https://docs.docker.com/}. Po spustení inštalátora sa Docker automaticky nainštaluje. Ďalej pokračujte podľa pokynov inštalátora.

    \item \textbf{Git}

    Zdrojový kód je dostupný na Git adrese: \url{https://github.com/Shotgunxq/bp}

    Git inštalátor si môžeme stiahnuť cez webovú stránku: \url{https://git-scm.com/downloads}

    Po spustení inštalátora sa Git automaticky nainštaluje.
\end{itemize}

Po nainštalovaní Gitu môžeme skopírovať zdrojový kód webovej aplikácie na naše zariadenie cez PowerShell pomocou príkazu:
\begin{verbatim}
git clone https://github.com/Shotgunxq/bp
\end{verbatim}

Po skopírovaní navigujte do tohto priečinka v PowerShell pomocou príkazu:
\begin{verbatim}
cd bp
\end{verbatim}

Zbuildujte aplikáciu pomocou príkazu:
\begin{verbatim}
docker compose build
\end{verbatim}
Počkajte, kým sa vytvoria Docker kontajnery, ktoré potom následne uvidíte v PowerShell alebo v Docker Desktop medzi \textit{Containers}.

Po úspešnom builde môžete spustiť aplikáciu pomocou príkazu:
\begin{verbatim}
docker compose up
\end{verbatim}

Aplikácia sa následne automaticky spustí. Databáza sa vytvorí spolu s databázovou schémou a hotovými dátami automaticky.

Aplikácia bude dostupná na adrese:
\begin{verbatim}
http://localhost/
\end{verbatim}