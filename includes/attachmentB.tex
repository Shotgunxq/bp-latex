Upozornenie: Tento návod na inštaláciu je platný výlučne pre operačný systém \textbf{Windows}. Postupy pre iné operačné systémy sa môžu líšiť.

Na úspešné spustenie webovej aplikácie je potrebné mať na vašom zariadení nainštalované nasledujúce komponenty:

\begin{itemize}
    \item \textbf{Windows Subsystem for Linux (WSL)}

    Oficiálny návod k inštalácii je dostupný na stránke:
    \url{https://learn.microsoft.com/en-us/windows/wsl/install}

    Otvorte Windows PowerShell pomocou klávesovej skratky \textbf{Windows Key + X}, ktorá otvorí menu. Kliknite na položku \textbf{Windows PowerShell}, čím otvoríte okno PowerShellu.

    V PowerShell zadajte príkaz na inštaláciu:
    \begin{verbatim}
    wsl --install
    \end{verbatim}

    Po dokončení automatickej inštalácie si vytvoríte lokálne konto pre subsystém podľa zobrazených pokynov.

    \item \textbf{Docker Desktop}

    Oficiálny návod na inštaláciu je dostupný na:
    \url{https://docs.docker.com/desktop/setup/install/windows-install/}

    Stiahnite inštalátor zo stránky:
    \url{https://docs.docker.com/}

    Po spustení inštalátora postupujte podľa pokynov. Docker Desktop sa nainštaluje automaticky.

    \item \textbf{Git (voliteľné)}

    Git je potrebný iba v prípade, že chcete zdrojový kód stiahnuť priamo z GitHubu. Ak využijete priložené súbory, Git nemusíte inštalovať.

    Inštalátor je dostupný na stránke:
    \url{https://git-scm.com/downloads}

    Po stiahnutí a spustení inštalátora postupujte podľa jeho pokynov.
\end{itemize}

\subsection*{Získanie zdrojového kódu}

Zdrojový kód aplikácie môžete získať dvoma spôsobmi:

\begin{enumerate}
    \item \textbf{Zo súboru priloženého k práci (odporúčané)}

    Ak používate súbor priložený k práci, je potrebné ho najskôr rozbaliť. Kliknite pravým tlačidlom myši na stiahnutý \texttt{.zip} súbor a vyberte možnosť \textbf{Extract All} (alebo \textbf{Extrahovať všetko}). Po rozbalení získate priečinok s aplikáciou.

    \item \textbf{Priamo z GitHubu}

    Ak využívate tento spôsob, najskôr si nainštalujte Git podľa pokynov vyššie.

    Potom otvorte PowerShell (\textbf{Windows Key + X}, položka \textbf{Windows PowerShell}) a stiahnite zdrojový kód príkazom:

    \begin{verbatim}
    git clone https://github.com/Shotgunxq/bp
    \end{verbatim}
\end{enumerate}

\subsection*{Spustenie aplikácie}

Bez ohľadu na spôsob získania zdrojového kódu pokračujte nasledovne:

Otvorte PowerShell (\textbf{Windows Key + X}, položka \textbf{Windows PowerShell}) a presuňte sa do priečinka aplikácie príkazom:

\begin{verbatim}
cd cesta\k\vasemu\priecinku\bp
\end{verbatim}

Cestu (\texttt{cesta\textbackslash k\textbackslash vasemu\textbackslash priecinku}) nahraďte úplnou cestou k priečinku, do ktorého ste aplikáciu rozbalili alebo stiahli.

Zbuildujte aplikáciu príkazom:

\begin{verbatim}
docker compose build
\end{verbatim}

Počkajte na dokončenie procesu. Docker kontajnery budú viditeľné v PowerShell alebo v Docker Desktop v sekcii \textit{Containers}. V tejto sekcii sa vám zobrazí tabuľka so zoznamom kontajnerov. V stĺpci \textbf{Action} kliknite na tlačidlo \textbf{Start}, aby ste kontajnery spustili manuálne.

Alternatívne môžete spustiť aplikáciu priamo z PowerShellu pomocou príkazu:

\begin{verbatim}
docker compose up
\end{verbatim}

Aplikácia automaticky vytvorí databázu, schému a importuje dáta.

Aplikácia bude dostupná vo vašom webovom prehliadači na adrese:

\begin{verbatim}
http://localhost/
\end{verbatim}