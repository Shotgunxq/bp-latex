\section{Analýza}
V tejto kapitole sa venujeme rozboru dostupných platforiem pre e-learning matematiky. 
Cieľom je identifikovať platformy, porovnať ich funkcie a odhaliť medzery, ktoré naša webová aplikácia môže vyplniť. 
Na trhu existuje široká škála platforiem pre e-learning rôzných matematických tém, 
z ktorých každá ponúka rôzne riešenia, funkcie a zameriava sa na odlišné cieľové skupiny.

\subsection{Brilliant.org}
Brilliant.org je online vzdelávacia platforma zameraná na interaktívne kurzy v oblastiach matematiky, vedy a počítačovej vedy.
Je navrhnutá tak, aby podporovala aktívne učenie prostredníctvom riešenia problémov a interaktívnych výziev, čím pomáha študentom rozvíjať kritické myslenie a logické schopnosti.
Platforma ponúka viac ako 60 kurzov, ktoré sú prispôsobené rôznym úrovniam znalostí, od začiatočníkov po pokročilých.

Medzi jej hlavné výhody patria interaktívne lekcie, ktoré sú navrhnuté tak, aby boli pútavé a vyžadovali aktívnu účasť študentov, čím zvyšujú efektivitu učenia.
Umožňuje tiež flexibilné a samostatné štúdium, čo je ideálne pre individuálne potreby. 
Platforma ponúka denné výzvy na rôzne témy, ktoré pomáhajú udržiavať študentov motivovaných a neustále zapojených do procesu učenia.
Nevýhodou je, že táto platforma je platená a dostupná len v anglickom jazyku, čo môže predstavovať prekážku pre niektorých študentov.
\cite{brilliant}
\subsection{Khan Academy}
Táto platforma ponúka bezplatné videokurzy a interaktívne cvičenia z rôznych oblastí matematiky, vrátane vysokoškolskej štatistiky a pravdepodobnosti.
Je vhodná pre študentov základných aj vysokých škôl. 
Medzi jej výhody patrí široká škála obsahu, jednoduché použitie a dostupnosť pre rôzne úrovne znalostí.
Dostupné zdroje k daným témam sú prehľadné a dobre štruktúrované.
Taktiež ponúka možnosť sledovania pokroku, získavania bodov a odznakov za splnené kapitoly, čím motivuje študentov k učeniu prostredníctvom gamifikácie.
Nevýhodou je, že je dostupná len v anglickom jazyku, čo môže byť pre niektorých študentov prekážkou. 
Používateľské rozhranie môže byť z dôvodu množstva obsahu pre niektorých používateľov neprehľadné, najmä ak sa na platforme nachádzajú prvýkrát. 
Napriek týmto nedostatkom je platforma považovaná za jeden z najlepších nástrojov na online vzdelávanie a sebarozvoj. \cite{khanacademy}
\subsection{Vieme matiku}
Najpopularnejším slovenským portálom pre e-learning matematiky je Vieme matiku.
Táto platforma ponúka rôzne kurzy a cvičenia z matematiky pre žiakov základných a stredných škôl.
Medzi jej výhody patrí dostupnosť pre slovenských žiakov, široký výber tém, rôzné formy precvičovania,
do ktorých patrí grafické znázornenie úloh a možnosť sledovania pokroku.
Ponúka taktiež hravé prvky, ako sú grafické a zvukové efekty, ktoré môžu zvýšiť motiváciu žiakov.
Vyznačuje sa taktiež jednoduchým použitím a prehľadným rozhraním.
Nevýhodou je, že nie je dostupná pre študentov mimo Slovenska, je podporovaná len v slovenčine.
Platforma služi na precvičovanie matematických úloh, ale neponúka zdroje pre samostatné štúdium alebo nápovedy. 
Zároveň, v prípade, že by sme chceli naplno využiť všetky jej funkcie, by bolo potrebné si zakúpiť licenciu. \cite{viemeto}
\subsection{Zhodnotenie}
Počas analýzy existujúcich vzdelávacích platforiem sme zistili, že na trhu chýbajú lokalizované a cenovo dostupné e-learningové riešenia pre stredoškolských a vysokoškolských študentov, ktoré by efektívne kombinovali gamifikáciu, interaktivitu a prehľadné rozhranie. Existujúce platformy, ako Brilliant.org a Khan Academy, ponúkajú kvalitné vzdelávacie materiály, ale ich dostupnosť je limitovaná anglickým jazykom a v prípade Brilliant.org aj plateným modelom. Vieme Matiku síce poskytuje lokalizovaný obsah, ale nezohľadňuje pokročilé potreby samostatného štúdia a je obmedzená na úzky okruh používateľov.

Analyzované platformy ukázali širokú škálu prístupov, pričom mnohé sa zameriavajú na riešenie komplexných úloh alebo tradičné formy vzdelávania. Tieto prístupy však často nekladú dôraz na intuitívne osvojovanie matematických konceptov a podporu samostatného učenia. Tieto poznatky nám umožňujú identifikovať medzery a formulovať jasné požiadavky na vývoj novej aplikácie, ktorá by ponúkala lokalizovaný obsah, interaktívne učenie a dostupnosť pre rôzne cieľové skupiny.
\begin{table}[htbp]
\caption{Vzdelávacie platformy}
\label{vzdelavaciePlatformy}
\begin{tabularx}{\textwidth}{|X|X|X|X|}
\hline
\textbf{Platforma} & \textbf{Funkcie} & \textbf{Cieľová skupina} & \textbf{Cena} \\ \hline
Khan Academy & Videokurzy, cvičenia & Všetky úrovne & Bezplatná \\ \hline
Brilliant.org & Gamifikované kurzy & Stredné a Vysoké školy & Platená \\ \hline
Vieme Matiku & Online kurzy matematiky & Základné a Stredné školy & Čiastočne bezplatná \\ \hline
\end{tabularx}
\end{table}

\section{Použité technológie a knižnice}
V tejto kapitole sa podrobne venujeme technológiám a knižniciam, ktoré plánujeme
použiť na vývoj webovej aplikácie pre e-learning matematickej štatistiky a
pravdepodobnosti. Výber technológií je založený na princípoch flexibility,
kompatibility, bezpečnosti a aktívnej komunity vývojárov.


\subsection{Frontend}
Frontend je časť softvérového vývoja, ktorá sa zaoberá tým, čo používateľ vidí a s čím interaguje pri práci s aplikáciou alebo webovou stránkou. 
Ide o viditeľnú vrstvu aplikácie, ktorá zahŕňa všetky prvky používateľského rozhrania (\acrshort{ui}) a je priamo zodpovedná za používateľskú skúsenosť (\acrshort{ux}).

V kontexte nášho webového vývoja predstavuje frontend technológie a nástroje používané na tvorbu webových stránok, ktoré sú dostupné a vykresľované v internetových prehliadačoch. 
Zahŕňa návrh, implementáciu a optimalizáciu používateľského rozhrania tak, aby bolo esteticky príťažlivé, funkčné a dostupné na rôznych zariadeniach a platformách.
\subsubsection{HTML}
\acrfull{html} je značkovací jazyk používaný na tvorbu a štruktúrovanie obsahu webových stránok.
Umožňuje definovať rôzne prvky, ako sú nadpisy, odseky, obrázky či odkazy, čím určuje základnú kostru a vzhľad webovej stránky.
Napriek častým mylným predstavám, HTML nie je programovací jazyk, keďže neumožňuje vytvárať podmienené logické operácie alebo funkcie.
Jeho hlavnou úlohou je prezentácia a organizácia obsahu pre webové prehliadače. 
\cite{HTML}

\subsubsection{CSS}
\acrfull{css} \cite{css} je štýlovací jazyk používaný na definovanie vzhľadu a formátovania webových stránok. 
Umožňuje oddeliť vizuálnu prezentáciu od štruktúry obsahu definovanej v \acrshort{html}, čím zjednodušuje údržbu a aktualizáciu dizajnu.
Pomocou \acrshort{css} je možné nastaviť rôzne vizuálne vlastnosti, ako sú farby, písma, veľkosti, rozloženie prvkov a ďalšie aspekty dizajnu.
 Taktiež podporuje tvorbu responzívnych dizajnov, ktoré sa prispôsobujú rôznym zariadeniam a veľkostiam obrazoviek. 
 Moderné techniky, ako flexbox a grid, umožňujú presné rozmiestnenie a zarovnanie prvkov na stránke, čo je užitočné pri tvorbe komplexných rozložení.

\subsubsection{SCSS}
\acrfull{scss} \cite{scss} je rozšírenie jazyka \acrshort{css}, ktoré pridáva pokročilé funkcie pre efektívnejšie štýlovanie webových stránok. 
\acrshort{scss} umožňuje používať premenné, vnáranie selektorov, mixiny, funkcie a operácie, čím zjednodušuje správu a údržbu štýlov.
 Vďaka svojim vlastnostiam podporuje modulárny prístup k tvorbe štýlov, čím zlepšuje čitateľnosť kódu a urýchľuje vývoj.

 \acrshort{scss} používa štandardnú \acrshort{css} syntax s doplnením nových funkcií, čo zabezpečuje spätnú kompatibilitu.
 Kód napísaný v \acrshort{scss} sa následne kompiluje do klasického \acrshort{css}, ktoré podporujú všetky moderné prehliadače. 
Tento proces zvyšuje flexibilitu vývoja a umožňuje tvorbu komplexných štýlových štruktúr.
\subsubsection{JavaScript}
JavaScript \cite{JavaScript} je interpretovaný programovací jazyk ktorý umožňuje dynamickú interakciu s používateľom a zmeny obsahu webových stránok bez nutnosti ich opätovného načítania.

Podporuje objektovo orientované programovanie s triedami, objektmi a metódami, čo umožňuje tvorbu komplexných aplikácií. Vďaka svojej dynamickej povahe dokáže meniť obsah a štruktúru stránky počas jej behu.

Medzi jeho funkcie patrí funkcionálne programovanie, kde sú funkcie považované za prvotriedne objekty, a programovanie riadené udalosťami, ktoré umožňuje reagovať na interakcie používateľa, napríklad na kliknutia.

Je multiplatformový a podporuje rôzne zariadenia, ako sú počítače, smartfóny a tablety. Populárne knižnice a rámce ako jQuery, React, Angular a Vue výrazne uľahčujú vývoj aplikácií.

Medzi hlavné vlastnosti patrí manipulácia s \acrfull{dom}, spracovanie udalostí, manipulácia s dátami a podpora asynchrónnych volaní na server pomocou techniky \acrfull{ajax}.
 Tieto vlastnosti z neho robia základný nástroj na tvorbu moderných webových aplikácií.
\subsubsection{TypeScript}
TypeScript \cite{TypeScript} je programovací jazyk vyvinutý spoločnosťou Microsoft, ktorý rozširuje možnosti JavaScriptu pridaním statického typovania a pokročilých objektovo orientovaných prvkov.
 Tým umožňuje vývojárom identifikovať chyby už počas vývoja, čo zvyšuje spoľahlivosť a udržiavateľnosť kódu.
  TypeScript je nadmnožinou JavaScriptu, čo znamená, že všetok platný kód v JavaScripte je kompatibilný s TypeScriptom.
   Po napísaní sa kód v TypeScripte transpiluje do štandardného JavaScriptu, ktorý je podporovaný vo všetkých moderných prehliadačoch. 
   Tento prístup umožňuje využívať výhody moderných programovacích techník pri zachovaní širokej kompatibility a flexibility, ktorú JavaScript ponúka.
\subsubsection{MathJax}
MathJax \cite{MathJax} je open-source JavaScriptový engine určený na zobrazovanie matematickej notácie, ako sú LaTeX, MathML a AsciiMath, v moderných webových prehliadačoch.
Je navrhnutý tak, aby konsolidoval pokroky vo webových technológiách do jednotnej platformy pre matematiku na webe, podporujúc hlavné prehliadače a operačné systémy, vrátane mobilných zariadení.
Používatelia nemusia inštalovať žiadne doplnky ani softvér; stačí, aby autor stránky zahrnul MathJax a matematický obsah do webovej stránky, a MathJax sa postará o zvyšok.

\subsection{UI a UX}
Používateľské rozhranie predstavuje vizuálnu časť digitálneho produktu, s ktorou používateľ priamo interaguje. 
Zahŕňa prvky ako tlačidlá, ikony, typografiu a farebné schémy. 
Cieľom UI dizajnu je vytvoriť esteticky príjemné a funkčné prostredie, ktoré uľahčuje používateľovi navigáciu a interakciu s produktom.

Používateľský zážitok sa zameriava na celkovú skúsenosť používateľa pri interakcii s produktom alebo službou. 
Ide o to, ako intuitívne a efektívne dokáže používateľ dosiahnuť svoje ciele. 
UX dizajn zahŕňa analýzu potrieb používateľov, návrh informačnej architektúry, prototypovanie a testovanie, s cieľom zabezpečiť, aby bol produkt nielen funkčný, ale aj príjemný na používanie.

Hoci sú UI a UX odlišné disciplíny, úzko spolupracujú s cieľom vytvoriť digitálne produkty, ktoré sú nielen vizuálne atraktívne, ale aj používateľsky prívetivé a efektívne.\cite{uiux}
\subsubsection{Figma}
Figma je cloudová platforma na návrh a prototypovanie používateľských rozhraní pre webové a mobilné aplikácie. 
Umožňuje tímovú spoluprácu v reálnom čase prostredníctvom webového prehliadača alebo desktopovej aplikácie. Podporuje vektorové ilustrácie, interaktívne prototypy a dizajnové systémy.
 Všetky projekty sú uložené v cloude, čo zjednodušuje prístup a zdieľanie. \cite{figma}
 \subsubsection{Material UI}
 \subsubsection{Angular}
%  \subsubsection{Bootstrap}

\subsection{Backend}
\subsubsection{Node.js}
\subsubsection{Express.js}
\subsubsection{CORS}
\subsection{Databázové systémy}
\subsubsection{PostgreSQL}
\subsubsection{DBDiagram}
\subsection{Framework}
\subsubsection{Frontendové frameworky}
\subsubsection{Backendové frameworky}

\subsection{Gamifikácia}
\subsubsection{Gamifikácia v e-learningu}
\subsection{Server}
\subsubsection{Kontejnerizácia}
\subsubsection{Docker}
\subsubsection{NGINX}
\subsection{GIT}