\section{Analýza}
V tejto kapitole sa podrobne venujeme rozboru dostupných platforiem pre e-learning matematickej štatistiky a pravdepodobnosti. Cieľom je identifikovať platformy, porovnať ich funkcie a odhaliť medzery na trhu, ktoré naša webová aplikácia môže vyplniť. Na trhu existuje široká škála platforiem pre e-learning matematickej štatistiky a pravdepodobnosti, z ktorých každá ponúka rôzne funkcie a zameriava sa na odlišné cieľové skupiny.
\subsection{Vieme matiku}
\subsection{Brilliant.org}
\subsection{Khan Academy}
Táto platforma ponúka bezplatné videokurzy a interaktívne cvičenia z rôznych oblastí
matematiky, vrátane štatistiky a pravdepodobnosti. Je vhodná pre študentov všetkých
úrovní, od začiatočníkov až po pokročilých. Medzi jej výhody patrí široká škála obsahu,
jednoduché použitie a dostupnosť pre rôzne úrovne znalostí. Nevýhodou je, že nie je
vždy personalizovaná, chýbajú interaktívne simulácie a gamifikácia a dostupná je len v
angličtine.