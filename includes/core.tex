\section{Analýza}
V tejto kapitole sa venujeme rozboru dostupných platforiem pre e-learning matematiky. 
Cieľom je identifikovať platformy, porovnať ich funkcie a odhaliť medzery, ktoré naša webová aplikácia môže vyplniť. 
Na trhu existuje široká škála platforiem pre e-learning rôzných matematických tém, 
z ktorých každá ponúka rôzne riešenia, funkcie a zameriava sa na odlišné cieľové skupiny.

\subsection{Brilliant.org}
Brilliant.org je online vzdelávacia platforma zameraná na interaktívne kurzy v oblastiach matematiky, vedy a počítačovej vedy.
Je navrhnutá tak, aby podporovala aktívne učenie prostredníctvom riešenia problémov a interaktívnych výziev, čím pomáha študentom rozvíjať kritické myslenie a logické schopnosti.
Platforma ponúka viac ako 60 kurzov, ktoré sú prispôsobené rôznym úrovniam znalostí, od začiatočníkov po pokročilých.

Medzi jej hlavné výhody patria interaktívne lekcie, ktoré sú navrhnuté tak, aby boli pútavé a vyžadovali aktívnu účasť študentov, čím zvyšujú efektivitu učenia.
Umožňuje tiež flexibilné a samostatné štúdium, čo je ideálne pre individuálne potreby. 
Platforma ponúka denné výzvy na rôzne témy, ktoré pomáhajú udržiavať študentov motivovaných a neustále zapojených do procesu učenia.
Nevýhodou je, že táto platforma je platená a dostupná len v anglickom jazyku, čo môže predstavovať prekážku pre niektorých študentov.
\cite{brilliant}
\subsection{Khan Academy}
Táto platforma ponúka bezplatné videokurzy a interaktívne cvičenia z rôznych oblastí matematiky, vrátane vysokoškolskej štatistiky a pravdepodobnosti.
Je vhodná pre študentov základných aj vysokých škôl. 
Medzi jej výhody patrí široká škála obsahu, jednoduché použitie a dostupnosť pre rôzne úrovne znalostí.
Dostupné zdroje k daným témam sú prehľadné a dobre štruktúrované.
Taktiež ponúka možnosť sledovania pokroku, získavania bodov a odznakov za splnené kapitoly, čím motivuje študentov k učeniu prostredníctvom gamifikácie\footnote{Gamifikácia je využitie herných prvkov a mechaník v nehernom prostredí s cieľom zvýšiť motiváciu, zapojenie a efektivitu používateľov.}.
Nevýhodou je, že je dostupná len v anglickom jazyku, čo môže byť pre niektorých študentov prekážkou. 
Používateľské rozhranie môže byť z dôvodu množstva obsahu pre niektorých používateľov neprehľadné, najmä ak sa na platforme nachádzajú prvýkrát. 
Napriek týmto nedostatkom je platforma považovaná za jeden z najlepších nástrojov na online vzdelávanie a sebarozvoj. \cite{khanacademy}
\subsection{Vieme matiku}
Najpopularnejším slovenským portálom pre e-learning matematiky je Vieme matiku.
Táto platforma ponúka rôzne kurzy a cvičenia z matematiky pre žiakov základných a stredných škôl.
Medzi jej výhody patrí dostupnosť pre slovenských žiakov, široký výber tém, rôzné formy precvičovania,
do ktorých patrí grafické znázornenie úloh a možnosť sledovania pokroku.
Ponúka taktiež hravé prvky, ako sú grafické a zvukové efekty, ktoré môžu zvýšiť motiváciu žiakov.
Vyznačuje sa taktiež jednoduchým použitím a prehľadným rozhraním.
Nevýhodou je, že nie je dostupná pre študentov mimo Slovenska, je podporovaná len v slovenčine.
Platforma služi na precvičovanie matematických úloh, ale neponúka zdroje pre samostatné štúdium alebo nápovedy. 
Zároveň, v prípade, že by sme chceli naplno využiť všetky jej funkcie, by bolo potrebné si zakúpiť licenciu. \cite{viemeto}
\subsection{Zhodnotenie}
Počas analýzy existujúcich vzdelávacích platforiem sme zistili, že na trhu chýbajú lokalizované a cenovo dostupné e-learningové riešenia pre stredoškolských a vysokoškolských študentov, ktoré by efektívne kombinovali gamifikáciu, interaktivitu a prehľadné rozhranie. 
Existujúce platformy, ako Brilliant.org a Khan Academy, ponúkajú kvalitné vzdelávacie materiály, ale ich dostupnosť je limitovaná anglickým jazykom a v prípade Brilliant.org aj plateným modelom. 
Vieme Matiku síce poskytuje lokalizovaný obsah, ale nezohľadňuje pokročilé potreby samostatného štúdia a je obmedzená na úzky okruh používateľov.
Analyzované platformy ukázali širokú škálu prístupov, pričom mnohé sa zameriavajú na riešenie komplexných úloh alebo tradičné formy vzdelávania.
 Tieto prístupy však často nekladú dôraz na intuitívne osvojovanie matematických konceptov a podporu samostatného učenia. 
Tieto poznatky nám umožňujú identifikovať medzery a formulovať jasné požiadavky na vývoj novej aplikácie, ktorá by ponúkala lokalizovaný obsah, interaktívne učenie a dostupnosť pre rôzne cieľové skupiny.
\begin{table}[htbp]
\caption{Vzdelávacie platformy}
\label{vzdelavaciePlatformy}
\begin{tabularx}{\textwidth}{|X|X|X|X|}
\hline
\textbf{Platforma} & \textbf{Funkcie} & \textbf{Cieľová skupina} & \textbf{Cena} \\ \hline
Khan Academy & Videokurzy, cvičenia & Všetky úrovne & Bezplatná \\ \hline
Brilliant.org & Gamifikované kurzy & Stredné a Vysoké školy & Platená \\ \hline
Vieme Matiku & Online kurzy matematiky & Základné a Stredné školy & Čiastočne bezplatná \\ \hline
\end{tabularx}
\end{table}

\section{Použité technológie a knižnice}
V tejto kapitole sa podrobne venujeme technológiám a knižniciam, ktoré plánujeme
použiť na vývoj webovej aplikácie pre e-learning matematickej štatistiky a
pravdepodobnosti. Výber technológií je založený na princípoch flexibility,
kompatibility, bezpečnosti a aktívnej komunity vývojárov.


\subsection{Frontend}
Frontend je časť softvérového vývoja, ktorá sa zaoberá tým, čo používateľ vidí a s čím interaguje pri práci s aplikáciou alebo webovou stránkou. 
Ide o viditeľnú vrstvu aplikácie, ktorá zahŕňa všetky prvky používateľského rozhrania (\acrshort{ui}) a je priamo zodpovedná za používateľskú skúsenosť (\acrshort{ux}).

V kontexte nášho webového vývoja predstavuje frontend technológie a nástroje používané na tvorbu webových stránok, ktoré sú dostupné a vykresľované v internetových prehliadačoch. 
Zahŕňa návrh, implementáciu a optimalizáciu používateľského rozhrania tak, aby bolo esteticky príťažlivé, funkčné a dostupné na rôznych zariadeniach a platformách.
\subsubsection{HTML}
\acrfull{html} je značkovací jazyk používaný na tvorbu a štruktúrovanie obsahu webových stránok.
Umožňuje definovať rôzne prvky, ako sú nadpisy, odseky, obrázky či odkazy, čím určuje základnú kostru a vzhľad webovej stránky.
Napriek častým mylným predstavám, HTML nie je programovací jazyk, keďže neumožňuje vytvárať podmienené logické operácie alebo funkcie.
Jeho hlavnou úlohou je prezentácia a organizácia obsahu pre webové prehliadače. 
\cite{HTML}

\subsubsection{CSS}
\acrfull{css} \cite{css} je štýlovací jazyk používaný na definovanie vzhľadu a formátovania webových stránok. 
Umožňuje oddeliť vizuálnu prezentáciu od štruktúry obsahu definovanej v \acrshort{html}, čím zjednodušuje údržbu a aktualizáciu dizajnu.
Pomocou \acrshort{css} je možné nastaviť rôzne vizuálne vlastnosti, ako sú farby, písma, veľkosti, rozloženie prvkov a ďalšie aspekty dizajnu.
 Taktiež podporuje tvorbu responzívnych dizajnov, ktoré sa prispôsobujú rôznym zariadeniam a veľkostiam obrazoviek. 
 Moderné techniky, ako flexbox a grid, umožňujú presné rozmiestnenie a zarovnanie prvkov na stránke, čo je užitočné pri tvorbe komplexných rozložení.

\subsubsection{SCSS}
\acrfull{scss} je rozšírenie jazyka \acrshort{css}, ktoré pridáva pokročilé funkcie pre efektívnejšie štýlovanie webových stránok. 
\acrshort{scss} umožňuje používať premenné, vnáranie selektorov, mixiny, funkcie a operácie, čím zjednodušuje správu a údržbu štýlov.
 Vďaka svojim vlastnostiam podporuje modulárny prístup k tvorbe štýlov, čím zlepšuje čitateľnosť kódu a urýchľuje vývoj.

 \acrshort{scss} používa štandardnú \acrshort{css} syntax s doplnením nových funkcií, čo zabezpečuje spätnú kompatibilitu.
 Kód napísaný v \acrshort{scss} sa následne kompiluje do klasického \acrshort{css}, ktoré podporujú všetky moderné prehliadače. 
Tento proces zvyšuje flexibilitu vývoja a umožňuje tvorbu komplexných štýlových štruktúr.\cite{scss}
\subsubsection{JavaScript}
JavaScript \cite{JavaScript} je interpretovaný programovací jazyk ktorý umožňuje dynamickú interakciu s používateľom a zmeny obsahu webových stránok bez nutnosti ich opätovného načítania.

Podporuje objektovo orientované programovanie s triedami, objektmi a metódami, čo umožňuje tvorbu komplexných aplikácií. Vďaka svojej dynamickej povahe dokáže meniť obsah a štruktúru stránky počas jej behu.

Medzi jeho funkcie patrí funkcionálne programovanie, kde sú funkcie považované za prvotriedne objekty, a programovanie riadené udalosťami, ktoré umožňuje reagovať na interakcie používateľa, napríklad na kliknutia.

Je multiplatformový a podporuje rôzne zariadenia, ako sú počítače, smartfóny a tablety. Populárne knižnice a rámce ako jQuery, React, Angular a Vue výrazne uľahčujú vývoj aplikácií.

Medzi hlavné vlastnosti patrí manipulácia s \acrfull{dom}, spracovanie udalostí, manipulácia s dátami a podpora asynchrónnych volaní na server pomocou techniky \acrfull{ajax}.
 Tieto vlastnosti z neho robia základný nástroj na tvorbu moderných webových aplikácií.
 \subsubsection{AJAX}
 AJAX (Asynchronous JavaScript and XML) je technológia, ktorá umožňuje webovým aplikáciám načítavať a odosielať dáta na pozadí bez nutnosti obnoviť celú stránku.
  Funguje na kombinácii známych technológií – HTML, CSS, JavaScript a XMLHttpRequest (alebo moderného fetch) – a výrazne zlepšuje interaktivitu a používateľskú skúsenosť. 
 AJAX sa využíva napríklad v ankete bez reloadu, našeptávačoch alebo moderných single-page aplikáciách.\cite{ajax}
\subsubsection{jQuery}
jQuery je open-source knižnica napísaná v JavaScripte, ktorá uľahčuje prácu s HTML dokumentom, manipuláciu s DOM, prácu s udalosťami, animáciami a technológiou AJAX.
 Bola vytvorená v roku 2006 s cieľom zjednodušiť vývoj v čase, keď neexistovala jednotná podpora v prehliadačoch.
  Aj napriek tomu, že je dnes považovaná za technicky zastaranú a datovo náročnú, stále sa používa na mnohých weboch vďaka svojej jednoduchosti a veľkému množstvu dostupných pluginov.\cite{jquery}


 \subsubsection{TypeScript}
TypeScript \cite{TypeScript} je programovací jazyk vyvinutý spoločnosťou Microsoft, ktorý rozširuje možnosti JavaScriptu pridaním statického typovania a pokročilých objektovo orientovaných prvkov.
 Tým umožňuje vývojárom identifikovať chyby už počas vývoja, čo zvyšuje spoľahlivosť a udržiavateľnosť kódu.
  TypeScript je nadmnožinou JavaScriptu, čo znamená, že všetok platný kód v JavaScripte je kompatibilný s TypeScriptom.
   Po napísaní sa kód v TypeScripte transpiluje do štandardného JavaScriptu, ktorý je podporovaný vo všetkých moderných prehliadačoch. 
   Tento prístup umožňuje využívať výhody moderných programovacích techník pri zachovaní širokej kompatibility a flexibility, ktorú JavaScript ponúka.
\subsubsection{MathJax}
MathJax \cite{MathJax} je open-source\footnote{Open-source je softvér, ktorého zdrojový kód je voľne dostupný, môže byť používaný, upravovaný a distribuovaný kýmkoľvek.} JavaScriptový engine určený na zobrazovanie matematickej notácie, ako sú LaTeX, MathML a AsciiMath, v moderných webových prehliadačoch.
Je navrhnutý tak, aby konsolidoval pokroky vo webových technológiách do jednotnej platformy pre matematiku na webe, podporujúc hlavné prehliadače a operačné systémy, vrátane mobilných zariadení.
Používatelia nemusia inštalovať žiadne doplnky ani softvér; stačí, aby autor stránky zahrnul MathJax a matematický obsah do webovej stránky, a MathJax sa postará o zvyšok.


  \subsubsection{MathQuill}
  MathQuill \cite{mathquill} je open-source webový editor matematických výrazov navrhnutý tak, aby umožňoval jednoduché a estetické zadávanie matematiky priamo v prehliadači. 
  Umožňuje používateľom zadávať a zobrazovať matematické výrazy v známej vizuálnej forme pomocou syntaxe podobnej LaTeXu.

Je obľúbený v edukatívnych a interaktívnych webových aplikáciách, pretože poskytuje dynamické pole, v ktorom možno editovať matematické vzorce s okamžitým vizuálnym výstupom. 
MathQuill podporuje rôzne režimy – ako zobrazenie statických výrazov, interaktívne polia pre vstup používateľa alebo čítanie/zápis LaTeX kódu.
\subsubsection{ApexCharts}
ApexCharts\cite{apexcharts} je open-source JavaScriptová knižnica určená na vizualizáciu dát v podobe interaktívnych grafov.
 Poskytuje širokú škálu grafických typov a umožňuje ich jednoduchú integráciu do webových aplikácií s podporou populárnych frameworkov ako React, Angular či Vue. 
Vďaka svojej flexibilite, responzívnosti a podpore dynamických dát sa často využíva pri tvorbe analytických nástrojov a interaktívnych dashboardov.

\subsection{Backend}
Backend predstavuje časť softvérovej aplikácie, ktorá nie je priamo prístupná používateľom; ide o serverovú stranu v klient-server architektúre.
 Zodpovedá za hlavnú funkcionalitu aplikácie, vrátane spracovania webových požiadaviek, manipulácie s dátami a ich ukladania v databázach. Backend spolupracuje s frontendom, ktorý tvorí prezentačnú vrstvu aplikácie, s cieľom zabezpečiť komplexnú používateľskú skúsenosť. 
 Dáta generované na backende sú odosielané na frontend, kde sú prezentované používateľovi.
  Hoci sú často backend a frontend vyvíjané oddelenými tímami, hranica medzi nimi môže byť nejasná, čo vedie k prístupu známemu ako full-stack\footnote{Full-stack vývoj je proces, kde vývojár pracuje na frontende aj backende, teda na používateľskom rozhraní aj serverovej logike.} development, kde programátori pracujú na oboch stranách aplikácie.
   V minulosti backend pozostával prevažne z jednoduchých serverových skriptov, avšak s rozvojom webových technológií sa využívajú pokročilé frameworky umožňujúce dynamickú generáciu obsahu.
 Efektívny backendový kód je kľúčový pre optimalizáciu výkonu aplikácie, minimalizáciu zaťaženia servera a databázy, a zabezpečenie rýchlej odozvy pre používateľov. \cite{backend} 
 \subsubsection{API}
 \acrfull{api} je rozhranie, ktoré umožňuje jednej aplikácii komunikovať s inou aplikáciou alebo systémom. 
 Umožňuje vývojárom využívať už existujúce funkcionality bez toho, aby ich museli znova vytvárať, čím urýchľuje a zjednodušuje vývoj softvéru.
  \acrshort{api} definuje spôsob výmeny dát medzi klientom a serverom – napríklad keď mobilná aplikácia zobrazuje údaje o počasí získané zo servera.\cite{api}
 \subsubsection{REST}
 \acrshort{rest} \acrshort{api} (\acrfull{rest} acrfull{api}) je typ webového \acrshort{api}, ktorý umožňuje aplikáciám komunikovať medzi sebou pomocou štandardných \acrshort{http} protokolov. 
 REST API využíva jednoduché operácie ako čítanie, zápis, úprava a mazanie dát prostredníctvom \acrshort{url} adries a HTTP metód.

Jeho výhodou je jednoduchá štruktúra, škálovateľnosť a nezávislosť medzi klientom a serverom – klient nemusí poznať vnútornú logiku servera a server nemusí vedieť nič o klientskom rozhraní.
 REST API je široko používané pri vývoji moderných webových a mobilných aplikácií, pretože umožňuje flexibilnú a efektívnu výmenu dát.
\subsubsection{CORS}
CORS (Cross-Origin Resource Sharing) predstavuje mechanizmus, ktorý umožňuje realizáciu HTTP požiadaviek medzi rôznymi doménami (tzv. cross-origin požiadavky).
 Ide o spôsob, ako prehliadače môžu bezpečne vykonávať požiadavky na zdroje umiestnené na iných serveroch, než odkiaľ pochádza pôvodná webová stránka. 
 V minulosti takéto požiadavky bránila politika rovnakého pôvodu (same-origin policy), ktorá obmedzovala skripty bežiace v prehliadači výhradne na komunikáciu so serverom tej istej domény.

Mechanizmus CORS funguje tak, že server explicitne definuje, kto môže vytvárať požiadavky na jeho \acrshort{api}
, aké typy požiadaviek sú povolené a aké HTTP metódy sú podporované.
 Toto sa realizuje prostredníctvom špeciálnych HTTP hlavičiek, ktoré umožňujú prehliadaču určiť, či konkrétna požiadavka medzi rôznymi doménami môže byť bezpečne vykonaná.
 \cite{CORS}
\subsubsection{Node.js}
Node.js je runtime\footnote{Runtime prostredie je prostredie poskytujúce aplikácii všetky potrebné zdroje, knižnice a služby, ktoré sú potrebné na jej spustenie a beh.}
prostredie umožňujúce použitie JavaScriptu na strane servera, pričom využíva flexibilitu a jednoduchosť tohto programovacieho jazyka.
 Výhodou JavaScriptu sú jeho pokročilé koncepty ako „first-class functions“ (funkcie prvej triedy) a closures, ktoré umožňujú vytváranie efektívnych webových aplikácií. 
 Hoci JavaScript býva kritizovaný za nespoľahlivosť, táto kritika vyplýva predovšetkým zo zvláštností \acrshort{dom}-u v prehliadačoch, nie zo samotného jazyka.
 \subsection{Databázové systémy}
 Databázový systém je softvérové riešenie, ktoré slúži na efektívne uchovávanie, správu a vyhľadávanie dát. 
 Umožňuje vytvárať databázy, ktoré obsahujú štruktúrované dáta usporiadané v tabuľkách, a poskytuje nástroje na ich úpravu, triedenie, filtrovanie a prezentáciu.
 
 Databáza samotná je organizovaný súbor údajov, často rozdelený do tabuliek, ktoré spolu súvisia a medzi ktorými môžu byť definované vzťahy. 
 Cieľom databázového systému je zabezpečiť integritu, konzistenciu a dostupnosť dát, pričom používateľ môže s dátami pracovať pomocou formulárov, dotazov či zostáv.\cite{databazovesystemy}
 
 \subsubsection{SQL}
 \acrfull{sql} je jazyk navrhnutý na prácu s dátami v relačných databázach, pričom umožňuje ich efektívne ukladanie, organizáciu a získavanie.
 V \acrshort{sql} sa údaje ukladajú do tabuliek, ktoré fungujú ako dvojrozmerné štruktúry zložené z riadkov a stĺpcov. 
 Každá tabuľka predstavuje určitý typ objektu (napr. zákazníci, produkty) a každý riadok v tabuľke predstavuje jeden záznam s konkrétnymi údajmi.
 
 Pomocou dotazov (queries) je možné z databázy získať presne tie údaje, ktoré používateľ potrebuje – napríklad filtrovať podľa podmienok, zoskupovať dáta alebo ich spájať z viacerých tabuliek.
 Dotazy sú základom práce s \acrshort{sql} a umožňujú analyzovať, transformovať a prezentovať uložené informácie v požadovanej forme.\cite{sql}
 
 \subsubsection{PostgreSQL}
 PostgreSQL je pokročilý open-source objektovo-relačný databázový systém (\acrshort{ordbms}), ktorý kombinuje tradičné vlastnosti relačných databáz so schopnosťou pracovať s objektovo orientovanými prvkami. 
 To znamená, že okrem práce s tabuľkami a vzťahmi (typických pre \acrfull{rdbms}) podporuje aj vlastné dátové typy, dedičnosť, či ukladanie komplexných štruktúr, ako sú napríklad JSON či geografické dáta.
 
 Ako \acrshort{ordbms}, PostgreSQL umožňuje definovať vlastné funkcie, procedúry a typy, čím ponúka vysokú mieru prispôsobiteľnosti pre náročné aplikácie. 
 Je navrhnutý s dôrazom na integritu dát, rozšíriteľnosť a štandardy \acrshort{sql}, a preto je široko používaný v podnikových riešeniach, výskumných systémoch aj moderných webových aplikáciách.\cite{postgre}
 \subsubsection{DBDiagram}
 dbdiagram.io je bezplatný online nástroj, ktorý umožňuje vývojárom a dátovým analytikom jednoducho vytvárať diagramy a databázových štruktúr pomocou jednoduchého textového zápisu.
 Jeho hlavnou výhodou je schopnosť rýchlo a efektívne navrhovať a vizualizovať databázové schémy, čo uľahčuje plánovanie a komunikáciu v tímoch.
 Používatelia môžu definovať tabuľky, stĺpce a vzťahy medzi nimi v jednoduchom jazyku, pričom nástroj automaticky generuje zodpovedajúci diagram.
 Okrem toho dbdiagram.io podporuje import a export SQL skriptov, čo umožňuje integráciu s existujúcimi databázami a uľahčuje migráciu alebo dokumentáciu databázových štruktúr.
 Vďaka týmto funkciám je dbdiagram.io obľúbeným nástrojom pre rýchly návrh databáz a spoluprácu na databázových projektoch.
 \subsection{Framework}
 Framework je softvérová platforma poskytujúca vývojárom preddefinovanú štruktúru a nástroje, ktoré urýchľujú a zjednodušujú proces vývoja aplikácií.
 Existujú rôzne typy frameworkov podľa oblasti použitia, napríklad webové, desktopové či mobilné.
 Používanie frameworkov prináša výhody ako rýchlejší vývoj, lepšiu kvalitu kódu a štandardizáciu práce vývojárov. \cite{framework}
 \subsubsection{Frontendové frameworky}
 Frontend framework je špecializovaný typ frameworku určený na vývoj používateľských rozhraní webových aplikácií.
 Na rozdiel od všeobecných frameworkov, ktoré môžu pokrývať celý vývojový cyklus aplikácie, frontend framework sa sústreďuje výhradne na klientskú časť – teda na to, čo používateľ vidí a s čím interaguje v prehliadači.
 Poskytuje nástroje a štruktúru na organizáciu komponentov, správu stavu aplikácie a manipuláciu s \acrshort{dom}-om, čím umožňuje tvorbu dynamických, responzívnych a udržiavateľných rozhraní.
 \subsubsection{Rozdiel medzi CSS a JavaScript (JS) frameworkmi}
 \begin{itemize}
  \item CSS frameworky (napr. Bootstrap, Tailwind CSS, Materialize) sú zamerané na vizuálnu stránku webu – poskytujú hotové štýly, rozloženia, komponenty (tlačidlá, formuláre, mriežky), ktoré zjednodušujú tvorbu konzistentného a estetického dizajnu. Ide predovšetkým o „vzhľad“ aplikácie.
  \item JS frameworky (napr. React, Angular, Vue.js) naopak poskytujú funkčnú logiku a štruktúru aplikácie. 
  Umožňujú pracovať s dátami, dynamicky meniť obsah stránky bez opätovného načítania (tzv. \acrshort{spa}), pracovať s API a definovať správanie komponentov.\cite{feframework}
\end{itemize}
\subsubsection{Angular}
Angular je open-source framework vyvinutý spoločnosťou Google, určený na tvorbu dynamických a responzívnych webových aplikácií.
Umožňuje vývojárom vytvárať aplikácie s bohatou funkcionalitou prostredníctvom komponentovo orientovanej architektúry, ktorá podporuje opätovné použitie kódu a zjednodušuje údržbu aplikácií.
Angular využíva TypeScript, nadmnožinu JavaScriptu, ktorá pridáva statické typovanie a ďalšie funkcie zlepšujúce vývojový proces. 
Medzi kľúčové vlastnosti Angularu patrí obojsmerná väzba dát, ktorá synchronizuje model a zobrazenie, a modulárny systém, ktorý umožňuje rozdelenie aplikácie na menšie, ľahko spravovateľné časti.
Angular tiež obsahuje nástroje na správu formulárov, komunikáciu so serverom a smerovanie, čo umožňuje vytvárať komplexné aplikácie s minimálnym úsilím. \cite{angular}

\subsubsection{RxJs}
\acrshort{rxjs} \acrfull{rxjs} je knižnica určená na reaktívne programovanie, ktorá umožňuje efektívne spracovanie asynchrónnych a udalostne založených dátových tokov pomocou tzv. observables. 
Tieto observables predstavujú prúdy dát (napr. kliknutia, HTTP odpovede), ktoré môžu byť pozorované a spracovávané rôznymi operátormi ako map, filter či reduce.

RxJS umožňuje prehľadnejšie a konzistentnejšie riadiť komplexné interakcie v aplikáciách, najmä tam, kde dochádza k častým zmenám stavu alebo udalostiam v čase.
Využíva sa najmä v moderných frontend frameworkoch, ako je Angular, kde zjednodušuje prácu s dátami a udalosťami.\cite{rxjs}
\subsubsection{Backendové frameworky}
Back-end framework je softvérový nástroj, ktorý uľahčuje vývoj serverovej časti webovej aplikácie. 
Poskytuje štruktúru, komponenty a funkcie potrebné na spracovanie požiadaviek, prácu s databázou, autentifikáciu a ďalšie serverové operácie. 
Cieľom back-end frameworku je zefektívniť vývoj, zabezpečiť konzistentnosť kódu a podporiť štandardizované riešenia. 
Rôzne frameworky (napr. Django, Spring, Laravel, Ruby on Rails, Express) sa líšia podľa náročnosti projektu, škálovateľnosti, dokumentácie a vhodnosti pre začiatočníkov či veľké podnikové aplikácie.\cite{backendframework}

Node.js tak využíva práve dobre definované vlastnosti JavaScriptu, ktoré umožňujú vytvárať vysoko výkonné platformy pre webové aplikácie.\cite{nodejs}
\subsubsection{Express.js}
Express je JavaScriptový framework fungujúci ako ľahká nadstavba nad Node.js, ktorý zjednodušuje tvorbu webových aplikácií a \acrshort{api}.
Zatiaľ čo Node.js poskytuje základné prostredie pre spúšťanie JavaScriptu na serveri, Express pridáva štruktúru a pomáha vývojárom vyhnúť sa opakovanému a zdĺhavému písaniu nízkoúrovňového kódu. 
Express je teda nástroj, ktorý zjednodušuje a zrýchľuje vývoj tým, že poskytuje vyššiu úroveň abstrakcie oproti základným funkciám dostupným v čistom Node.js.

Jeho minimalistická architektúra ponúka veľkú mieru flexibility – vývojári nie sú nútení dodržiavať presne stanovenú štruktúru projektu, no zároveň majú k dispozícii nástroje na efektívne spracovanie HTTP požiadaviek, tvorbu routovacích pravidiel a využívanie middleware vrstiev. 
Express umožňuje jednoduché vytváranie \acrshort{rest}ful \acrshort{api} a dynamických webových aplikácií.
Pre svoju jednoduchosť, rozsiahlu dokumentáciu a silnú komunitu patrí medzi najpopulárnejšie riešenia pre serverovú časť aplikácií postavených na JavaScripte.\cite{express}\cite{backendframework}
\subsection{UI a UX}
Používateľské rozhranie predstavuje vizuálnu časť digitálneho produktu, s ktorou používateľ priamo interaguje. 
Zahŕňa prvky ako tlačidlá, ikony, typografiu a farebné schémy. 
Cieľom UI dizajnu je vytvoriť esteticky príjemné a funkčné prostredie, ktoré uľahčuje používateľovi navigáciu a interakciu s produktom.

Používateľský zážitok sa zameriava na celkovú skúsenosť používateľa pri interakcii s produktom alebo službou. 
Ide o to, ako intuitívne a efektívne dokáže používateľ dosiahnuť svoje ciele. 
UX dizajn zahŕňa analýzu potrieb používateľov, návrh informačnej architektúry, prototypovanie a testovanie, s cieľom zabezpečiť, aby bol produkt nielen funkčný, ale aj príjemný na používanie.

Hoci sú UI a UX odlišné disciplíny, úzko spolupracujú s cieľom vytvoriť digitálne produkty, ktoré sú nielen vizuálne atraktívne, ale aj používateľsky prívetivé a efektívne.\cite{uiux}
\subsubsection{Figma}
Figma je cloudová platforma na návrh a prototypovanie používateľských rozhraní pre webové a mobilné aplikácie. 
Umožňuje tímovú spoluprácu v reálnom čase prostredníctvom webového prehliadača alebo desktopovej aplikácie. Podporuje vektorové ilustrácie, interaktívne prototypy a dizajnové systémy.
 Všetky projekty sú uložené v cloude, čo zjednodušuje prístup a zdieľanie. \cite{figma}
 \subsubsection{Material UI}
 Angular Material UI je oficiálna knižnica komponentov pre Angular, ktorá implementuje dizajnové princípy Material Design od spoločnosti Google. 
 Poskytuje predpripravené a prispôsobiteľné komponenty, ako sú tlačidlá, formuláre, tabuľky či navigačné panely, ktoré umožňujú vývojárom rýchlo vytvárať moderné, responzívne a esteticky príjemné používateľské rozhrania.
  Angular Material UI podporuje integráciu s Angular frameworkom a zabezpečuje konzistentný dizajn a vysokú úroveň použiteľnosti v rámci aplikácií. \cite{materialUI}

\subsection{Gamifikácia}
Gamifikácia je aplikácia herných prvkov a princípov v neherných kontextoch s cieľom zvýšiť angažovanosť a motiváciu jednotlivcov pri vykonávaní určitých aktivít.
 Tento prístup sa využíva v rôznych oblastiach, ako sú vzdelávanie, marketing, podnikanie a osobný rozvoj.
Vo vzdelávacom prostredí to znamená implementáciu mechanizmov, ako sú zbieranie bodov, získavanie odmien, porovnávanie sa s ostatnými či postupovanie na vyššie úrovne, aby sa proces učenia stal interaktívnejším a pútavejším. 
\subsubsection{Gamifikácia v e-learningu}
Gamifikácia v e-learningu zahŕňa integráciu herných prvkov, ako sú body, odznaky, rebríčky a úrovne, do vzdelávacích online prostredí s cieľom zvýšiť motiváciu a angažovanosť študentov.
 Podľa príspevku v zborníku z medzinárodnej vedeckej konferencie "Vzdělávání dospělých 2021" \cite{gamifikacia} je gamifikácia efektívnym nástrojom na podporu aktívneho učenia, pretože využíva prirodzenú ľudskú tendenciu k hre a súťaživosti.
  Implementácia herných mechanizmov v e-learningových kurzoch môže viesť k zlepšeniu zapojenia študentov, zvýšeniu ich motivácie a následne k lepším vzdelávacím výsledkom.

\subsection{Server}
Server je centrálny počítač v sieti, ktorý poskytuje služby a zdroje iným zariadeniam – tzv. klientom. 
Umožňuje uchovávať a sprístupňovať dáta, spúšťať aplikácie, spracovávať požiadavky používateľov alebo poskytovať webové stránky či e-mailové služby.
 Môže ísť o fyzický počítač alebo virtuálny stroj a jeho konkrétna funkcia závisí od typu – napríklad súborový, databázový, aplikačný alebo webový server. 
 Správne nastavený server je základom spoľahlivej a bezpečnej IT infraštruktúry.\cite{server}
\subsubsection{NGINX}
Nginx\cite{nginx} je výkonný a spoľahlivý webový a reverzný proxy server, ktorý je navrhnutý na efektívne spracovanie veľkého množstva súčasných pripojení pri minimálnom využití systémových zdrojov. 
Najčastejšie sa používa na poskytovanie statického obsahu, ako sú HTML, CSS, JavaScript alebo obrázky, a zároveň ako sprostredkovateľ medzi klientom a backend aplikáciou, čím preposiela požiadavky na servery bežiace napríklad v Node.js alebo PHP. 
Nginx dokáže tiež rozdeľovať záťaž medzi viaceré servery, čím zabezpečuje vyššiu dostupnosť a škálovateľnosť aplikácií, a môže slúžiť ako cache server pre rýchlejšie doručovanie často používaného obsahu.
 Vďaka svojej asynchrónnej architektúre a podpore moderných protokolov, ako sú \acrshort{http}/3 a \acrshort{tls}, je ideálnym riešením pre hosting moderných webových aplikácií s vysokými nárokmi na výkon.
\subsubsection{Docker}
Docker je open-source platforma, ktorá umožňuje vývoj, distribúciu a správu aplikácií pomocou tzv. kontajnerov. 
Kontajner je ľahké a izolované prostredie, ktoré obsahuje všetko potrebné na beh aplikácie – vrátane kódu, knižníc, runtime prostredia a nastavení.

Vďaka tomu môže aplikácia bežať rovnakým spôsobom na rôznych miestach, napríklad na vývojárskom počítači, v testovacom prostredí aj na produkčnom serveri. 
Docker využíva virtualizáciu na úrovni operačného systému, čo mu umožňuje spúšťať viacero kontajnerov efektívne a s nízkymi nárokmi na systémové prostriedky.\cite{docker}
\subsection{GIT}
Git je distribuovaný systém na správu verzií, ktorý umožňuje sledovať a zaznamenávať zmeny v súboroch počas vývoja softvéru.
 Umožňuje viacerým vývojárom pracovať súčasne na tom istom projekte, pričom každá kópia projektu obsahuje úplnú históriu zmien. 
 Git je známy svojou rýchlosťou, flexibilitou a podporou nelineárnych pracovných tokov. 
Je open-source, bezplatný a vďaka svojim vlastnostiam sa stal štandardom v oblasti verzionovania kódu.\cite{git}

\section{Návrh webovej aplikácie}
Webová aplikácia je koncipovaná ako dvojvrstvový systém, pozostávajúci z frontendovej časti, implementovanej pomocou frameworku Angular, a backendovej časti, vytvorenej pomocou platformy Express.
 Obe časti sú navrhnuté ako samostatné a na sebe nezávislé aplikácie, čo zvyšuje flexibilitu a umožňuje ich samostatný vývoj, testovanie a nasadzovanie.

 Backendová časť je zodpovedná za spracovanie údajov, manipuláciu s databázou, \acrshort{ldap} autentifikáciu používateľov a spracovanie požiadaviek zo strany klienta.
 Slúži ako integračná vrstva medzi používateľským rozhraním a databázou, pričom zabezpečuje bezpečný a efektívny prenos údajov. 
 Backend sme implementovali pomocou frameworku Express.js, ktorý je postavený na Node.js a umožňuje jednoducho vytvárať REST API rozhrania. 
 Medzi jeho hlavné výhody patrí jednoduchosť, flexibilita, rozsiahla komunita a množstvo dostupných middleware modulov, čo výrazne urýchľuje vývoj serverovej logiky.

Express.js je zároveň ideálne riešenie pre aplikácie typu \acrshort{spa}, akú predstavuje aj náš frontend vytvorený v Angulari. 
V tomto architektonickom modeli Express plní úlohu backendovej vrstvy, ktorá poskytuje API endpointy, prostredníctvom ktorých frontend získava dynamicky obsah a dáta. 
Táto kombinácia umožňuje rýchlu a plynulú interakciu používateľa s aplikáciou bez potreby neustáleho obnovovania stránky, čím sa zvyšuje používateľský komfort a celkový výkon systému.

Pomocou \acrshort{ldap} autentifikácie dokážeme navyše overovať identitu používateľov a priraďovať im prístupové práva na základe ich roly – napríklad ako študent alebo administrátor (učiteľ).
 Tento prístup umožňuje efektívne riadenie oprávnení a zabezpečenie rôznych úrovní prístupu k funkcionalitám aplikácie.

Frontendová časť aplikácie je navrhnutá ako \acrfull{spa}.
 Použitie Angular frameworku bolo motivované jeho výbornou škálovateľnosťou a komponentovým prístupom, ktorý umožňuje vytvárať zapuzdrené (encapsulované) komponenty.
  Tento prístup vedie k lepšej organizácii a čitateľnosti kódu, a zároveň uľahčuje jeho údržbu a opätovné použitie.

Komunikácia medzi frontendom a backendom je sprostredkovaná prostredníctvom tzv. servisných tried (Services), ktoré Angular poskytuje ako jednu zo svojich kľúčových funkcionalít.
 Tieto služby slúžia na získavanie údajov zo servera, validáciu používateľských vstupov, uchovávanie stavu aplikácie, ako aj na centrálne spracovanie logiky a dát. 
 Každá Angular služba je singleton\footnote{Návrhový vzor, ktorý zabezpečuje, že z určitej triedy existuje v aplikácii práve jedna inštancia, ktorá je globálne dostupná.} objekt, čo umožňuje zdieľanie dát a funkcionality naprieč celou aplikáciou. 
 Tento prístup významne zvyšuje modularitu systému a podporuje znovupoužiteľnosť kódu.

 Pri asynchrónnej komunikácii a práci s údajmi využívame v Angulari mechanizmus Observables, ktorý je súčasťou knižnice \acrshort{rxjs}. 
 Observables umožňujú efektívne pracovať s dátovými tokmi, ako sú napríklad \acrshort{http} požiadavky alebo používateľské vstupy, pričom poskytujú pokročilé možnosti ako je reaktívne spracovanie dát, zrušenie požiadaviek, spájanie viacerých streamov alebo transformácia údajov. 
 Vďaka tejto reaktívnej paradigme dokáže aplikácia flexibilne reagovať na zmeny v dátach a zvyšuje sa jej interaktivita, výkon a stabilita.