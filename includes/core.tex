\section{Analýza}
V tejto kapitole sa venujeme rozboru dostupných platforiem pre e-learning matematiky. 
Cieľom je identifikovať platformy, porovnať ich funkcie a odhaliť medzery, ktoré naša webová aplikácia môže vyplniť. 
Na trhu existuje široká škála platforiem pre e-learning rôzných matematických tém, 
z ktorých každá ponúka rôzne riešenia, funkcie a zameriava sa na odlišné cieľové skupiny.

\subsection{Brilliant.org}
Brilliant.org je online vzdelávacia platforma zameraná na interaktívne kurzy v oblastiach matematiky, vedy a počítačovej vedy.
Je navrhnutá tak, aby podporovala aktívne učenie prostredníctvom riešenia problémov a interaktívnych výziev, čím pomáha študentom rozvíjať kritické myslenie a logické schopnosti.
Platforma ponúka viac ako 60 kurzov, ktoré sú prispôsobené rôznym úrovniam znalostí, od začiatočníkov po pokročilých.

Medzi jej hlavné výhody patria interaktívne lekcie, ktoré sú navrhnuté tak, aby boli pútavé a vyžadovali aktívnu účasť študentov, čím zvyšujú efektivitu učenia.
Umožňuje tiež flexibilné a samostatné štúdium, čo je ideálne pre individuálne potreby. 
Platforma ponúka denné výzvy na rôzne témy, ktoré pomáhajú udržiavať študentov motivovaných a neustále zapojených do procesu učenia.
Nevýhodou je, že táto platforma je platená a dostupná len v anglickom jazyku, čo môže predstavovať prekážku pre niektorých študentov.
\cite{brilliant}
\subsection{Khan Academy}
Táto platforma ponúka bezplatné videokurzy a interaktívne cvičenia z rôznych oblastí matematiky, vrátane vysokoškolskej štatistiky a pravdepodobnosti.
Je vhodná pre študentov základných aj vysokých škôl. 
Medzi jej výhody patrí široká škála obsahu, jednoduché použitie a dostupnosť pre rôzne úrovne znalostí.
Dostupné zdroje k daným témam sú prehľadné a dobre štruktúrované.
Taktiež ponúka možnosť sledovania pokroku, získavania bodov a odznakov za splnené kapitoly, čím motivuje študentov k učeniu prostredníctvom gamifikácie.
Nevýhodou je, že je dostupná len v anglickom jazyku, čo môže byť pre niektorých študentov prekážkou. 
Používateľské rozhranie môže byť z dôvodu množstva obsahu pre niektorých používateľov neprehľadné, najmä ak sa na platforme nachádzajú prvýkrát. 
Napriek týmto nedostatkom je platforma považovaná za jeden z najlepších nástrojov na online vzdelávanie a sebarozvoj. \cite{khanacademy}
\subsection{Vieme matiku}
Najpopularnejším slovenským portálom pre e-learning matematiky je Vieme matiku.
Táto platforma ponúka rôzne kurzy a cvičenia z matematiky pre žiakov základných a stredných škôl.
Medzi jej výhody patrí dostupnosť pre slovenských žiakov, široký výber tém, rôzné formy precvičovania,
do ktorých patrí grafické znázornenie úloh a možnosť sledovania pokroku.
Ponúka taktiež hravé prvky, ako sú grafické a zvukové efekty, ktoré môžu zvýšiť motiváciu žiakov.
Vyznačuje sa taktiež jednoduchým použitím a prehľadným rozhraním.
Nevýhodou je, že nie je dostupná pre študentov mimo Slovenska, je podporovaná len v slovenčine.
Platforma služi na precvičovanie matematických úloh, ale neponúka zdroje pre samostatné štúdium alebo nápovedy. 
Zároveň, v prípade, že by sme chceli naplno využiť všetky jej funkcie, by bolo potrebné si zakúpiť licenciu. \cite{viemeto}
\subsection{Zhodnotenie}
Počas analýzy existujúcich vzdelávacích platforiem sme zistili, že na trhu chýbajú lokalizované a cenovo dostupné e-learningové riešenia pre stredoškolských a vysokoškolských študentov, ktoré by efektívne kombinovali gamifikáciu, interaktivitu a prehľadné rozhranie. Existujúce platformy, ako Brilliant.org a Khan Academy, ponúkajú kvalitné vzdelávacie materiály, ale ich dostupnosť je limitovaná anglickým jazykom a v prípade Brilliant.org aj plateným modelom. Vieme Matiku síce poskytuje lokalizovaný obsah, ale nezohľadňuje pokročilé potreby samostatného štúdia a je obmedzená na úzky okruh používateľov.

Analyzované platformy ukázali širokú škálu prístupov, pričom mnohé sa zameriavajú na riešenie komplexných úloh alebo tradičné formy vzdelávania. Tieto prístupy však často nekladú dôraz na intuitívne osvojovanie matematických konceptov a podporu samostatného učenia. Tieto poznatky nám umožňujú identifikovať medzery a formulovať jasné požiadavky na vývoj novej aplikácie, ktorá by ponúkala lokalizovaný obsah, interaktívne učenie a dostupnosť pre rôzne cieľové skupiny.
\begin{table}[htbp]
\caption{Vzdelávacie platformy}
\label{vzdelavaciePlatformy}
\begin{tabularx}{\textwidth}{|X|X|X|X|}
\hline
\textbf{Platforma} & \textbf{Funkcie} & \textbf{Cieľová skupina} & \textbf{Cena} \\ \hline
Khan Academy & Videokurzy, cvičenia & Všetky úrovne & Bezplatná \\ \hline
Brilliant.org & Gamifikované kurzy & Stredné a vysoké školy & Platená \\ \hline
Vieme Matiku & Online kurzy matematiky & Základné a Stredné školy & Čiastočne bezplatná \\ \hline
\end{tabularx}
\end{table}

\section{Použité technológie a knižnice}
V tejto kapitole sa podrobne venujeme technológiám a knižniciam, ktoré plánujeme
použiť na vývoj webovej aplikácie pre e-learning matematickej štatistiky a
pravdepodobnosti. Výber technológií je založený na princípoch flexibility,
kompatibility, bezpečnosti a aktívnej komunity vývojárov.
\subsection{Frontend}
\subsubsection{HTML}
\subsubsection{CSS}
\subsubsection{SCSS}
\subsubsection{JavaScript}
\subsubsection{TypeScript}

\subsection{Backend}
\subsubsection{Node.js}
\subsubsection{Express.js}
\subsubsection{CORS}
\subsection{Databázové systémy}
\subsubsection{PostgreSQL}
\subsubsection{DBDiagram}
\subsection{Framework}
\subsubsection{Frontendové frameworky}
\subsubsection{Backendové frameworky}
\subsection{UI a UIX}
\subsubsection{Bootstrap}
\subsubsection{Material UI}
\subsubsection{Angular}
\subsection{Gamifikácia}
\subsubsection{Gamifikácia v e-learningu}
\subsection{Server}
\subsubsection{Kontejnerizácia}
\subsubsection{Docker}
\subsubsection{NGINX}
\subsection{GIT}