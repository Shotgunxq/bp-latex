Štúdium matematickej štatistiky a pravdepodobnosti je signifikantným krokom k porozumeniu a analýze náhodných javov a dát v rôznych oblastiach, akými sú napríklad ekonómia, vedecký výskum alebo medicína.
Toto štúdium človeka núti nahliadnuť do širokého spektra matematických tém – či už ide o modus, medián, stredovú hodnotu, náhodné premenné, kombinatoriku alebo podmienenú pravdepodobnosť – pričom každá z týchto tém disponuje unikátnymi nástrojmi na kvantifikáciu a analýzu dátových súborov.

S cieľom podporiť systematické a efektívne vzdelávanie v týchto kľúčových oblastiach sme sa rozhodli navrhnúť a implementovať trojvrstvovú webovú aplikáciu dostupnú pomocou moderných technológií.
Frontendová časť aplikácie bola vyvinutá pomocou frameworku Angular v kombinácii s knižnicou Material UI, zatiaľ čo backend využíva Node.js s frameworkom Express.js.
Táto aplikácia využíva interaktívne a dynamické metódy učenia, ako sú interaktívne grafy, automaticky generované testy a moderné gamifikačné prvky, čím zlepšuje angažovanosť študentov a podporuje jednoduchšie osvojenie si matematických konceptov.

Cieľom práce je nielen vytvoriť intuitívny a dostupný nástroj na učenie sa matematickej štatistiky a pravdepodobnosti, ale aj zvýšiť prístupnosť a zrozumiteľnosť týchto náročných tém pre študentov. 
Aplikácia má ambíciu výrazne prispieť k efektívnejšiemu vzdelávaciemu procesu a rozvíjať analytické schopnosti a matematické myslenie používateľov prostredníctvom moderných a interaktívnych výučbových metód.