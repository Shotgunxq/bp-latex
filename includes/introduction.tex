Štúdium matematickej štatistiky a teórie pravdepodobnosti zohráva významnú úlohu pri porozumení a analýze náhodných javov a dát v rozličných oblastiach, akými sú ekonómia, vedecký výskum alebo medicína. 
Tieto matematické disciplíny zahŕňajú široké spektrum tém, medzi ktoré patrí analýza údajov pomocou modusu, mediánu či stredovej hodnoty, práca s náhodnými premennými, kombinatorika, alebo podmienená pravdepodobnosť. 
Každá z týchto oblastí poskytuje unikátne nástroje na kvantifikáciu, interpretáciu a predikciu rôznych javov a trendov.

S cieľom podporiť systematické a efektívne vzdelávanie v týchto kľúčových oblastiach sme sa rozhodli navrhnúť a implementovať trojvrstvovú webovú aplikáciu dostupnú pomocou moderných technológií.
Frontendová časť aplikácie bola vyvinutá pomocou frameworku Angular v kombinácii s knižnicou Material UI, zatiaľ čo backend využíva Node.js s frameworkom Express.js.
Táto aplikácia využíva interaktívne a dynamické metódy učenia, ako sú interaktívne grafy, automaticky generované testy a moderné gamifikačné prvky, čím zlepšuje angažovanosť študentov a podporuje jednoduchšie osvojenie si matematických konceptov.

Cieľom práce je nielen vytvoriť intuitívny a dostupný nástroj na učenie sa matematickej štatistiky a pravdepodobnosti, ale aj zvýšiť prístupnosť a zrozumiteľnosť týchto náročných tém pre študentov. 
Aplikácia má ambíciu výrazne prispieť k efektívnejšiemu vzdelávaciemu procesu a rozvíjať analytické schopnosti a matematické myslenie používateľov prostredníctvom moderných a interaktívnych výučbových metód.