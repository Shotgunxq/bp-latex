Cieľom tejto bakalárskej práce bolo navrhnúť a implementovať trojvrstvovú webovú aplikáciu, ktorá podporuje systematické vzdelávanie v oblasti matematickej štatistiky a pravdepodobnosti. Pri realizácii práce sme úspešne splnili všetky stanovené požiadavky, pričom sme vychádzali z odporúčanej odbornej literatúry a zároveň využili moderné technológie a vývojové postupy.

Výsledkom je plne funkčná a dostupná webová aplikácia, ktorá kombinuje interaktívne výučbové materiály, automaticky generované testy, vizualizáciu výsledkov a prvky gamifikácie. Táto kombinácia umožňuje študentom efektívnejšie si osvojovať teoretické poznatky a zároveň si ich overovať v praktických úlohách.

Súčasťou vývoja bolo aj testovanie aplikácie reálnymi používateľmi, vďaka ktorému sme získali cennú spätnú väzbu. Na základe tejto spätnej väzby sme upravili viaceré prvky používateľského rozhrania a zjednodušili niektoré interakcie, čo viedlo k zlepšeniu používateľskej skúsenosti a intuitívnosti práce s aplikáciou.

Aplikácia tak úspešne napĺňa svoj účel ako moderný e-learningový nástroj a predstavuje plnohodnotné riešenie v kontexte podpory výučby pravdepodobnosti a štatistiky.